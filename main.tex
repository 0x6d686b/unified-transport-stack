\documentclass[twocolumn,english]{IEEEtran}
\usepackage[T1]{fontenc}
\usepackage{babel}
\usepackage{amsthm}
\usepackage{graphicx}
\usepackage[unicode=true,
 bookmarks=true,bookmarksnumbered=true,bookmarksopen=true,bookmarksopenlevel=1,
 breaklinks=false,pdfborder={0 0 0},backref=false,colorlinks=false]
 {hyperref}
\hypersetup{pdftitle={Unified Transport Stack for Cloud Computing},
 pdfauthor={Mathias Habl``utzel},
 pdfpagelayout=OneColumn, pdfnewwindow=true, pdfstartview=XYZ, plainpages=false}

\makeatletter

%%%%%%%%%%%%%%%%%%%%%%%%%%%%%% LyX specific LaTeX commands.
\DeclareRobustCommand*{\lyxarrow}{%
\@ifstar
{\leavevmode\,$\triangleleft$\,\allowbreak}
{\leavevmode\,$\triangleright$\,\allowbreak}}
%% Because html converters don't know tabularnewline
\providecommand{\tabularnewline}{\\}
%% A simple dot to overcome graphicx limitations
\newcommand{\lyxdot}{.}


%%%%%%%%%%%%%%%%%%%%%%%%%%%%%% Textclass specific LaTeX commands.
 % protect \markboth against an old bug reintroduced in babel >= 3.8g
 \let\oldforeign@language\foreign@language
 \DeclareRobustCommand{\foreign@language}[1]{%
   \lowercase{\oldforeign@language{#1}}}
\theoremstyle{plain}
\newtheorem{thm}{\protect\theoremname}
\theoremstyle{plain}
\newtheorem{lem}[thm]{\protect\lemmaname}

%%%%%%%%%%%%%%%%%%%%%%%%%%%%%% User specified LaTeX commands.
% for subfigures/subtables
\ifCLASSOPTIONcompsoc
\usepackage[caption=false,font=normalsize,labelfont=sf,textfont=sf]{subfig}
\else
\usepackage[caption=false,font=footnotesize]{subfig}
\fi
\makeatother

\providecommand{\lemmaname}{Lemma}
\providecommand{\theoremname}{Theorem}

\begin{document}





\title{Unified Transport Stack for Cloud Computing}


\author{Mathias Habl''utzel, Second~Name,
\thanks{Z, ... Institute of ..., City,
Country, e-mail: \protect\href{http://xxx@xxx.xxx}{xxx@xxx.xxx}.%
}%
\thanks{Second~Name is with the Department of ..., ... Institute of ...,
City, Country, e-mail: \protect\href{http://xxx@xxx.xxx}{xxx@xxx.xxx}.%
}%
\thanks{Third~Name is with the Department of ..., ... Institute of ..., City,
Country, e-mail: \protect\href{http://xxx@xxx.xxx}{xxx@xxx.xxx}.%
}}


\IEEEspecialpapernotice{Invited Paper}


\IEEEaftertitletext{after title text like dedication}


\markboth{Journal of XXX}{Your Name \MakeLowercase{\emph{et al.}}: Your Title}


\IEEEpubid{0000--0000/00\$00.00~\copyright~2007 IEEE}
\maketitle
\begin{abstract}
This is the abstract text.\end{abstract}
\begin{IEEEkeywords}
simplicity, beauty, elegance
\end{IEEEkeywords}

\section{Introduction}



\IEEEPARstart{H}{ere} is the text text text text text text text
text text text text text text text text text.


\section{Previous Work}

text text text text text text text text text text text text text text
text


\subsection{subsection}


\subsection{another subsection}


\section{Methodology}
\begin{thm}[Theorem name]
For a named theorem or theorem-like environment you need to insert
the name through \textsf{Insert\lyxarrow{}Short Title}, as done here.\end{thm}
\begin{lem}
If you don't want a theorem or lemma name don't add one.\end{lem}
\begin{IEEEproof}
And here's the proof!
\end{IEEEproof}

\section{Results}

\begin{figure}[htbp]
\begin{centering}
\textsf{A single column figure goes here}
\par\end{centering}

\caption{Captions go \emph{under} the figure}
\end{figure}
\begin{table}[htbp]
\caption{Table captions go \emph{above} the table}


\centering{}%
\begin{tabular}{|c|c|}
\hline 
delete & this\tabularnewline
\hline 
\hline 
example & table\tabularnewline
\hline 
\end{tabular}
\end{table}



\section{Conclusions}

bla bla


\appendices{}


\section{First appendix}

Citation: \cite{IEEEexample:beebe_archive}


\section{Second appendix}


\section*{Acknowlegment}

bla bla

\bibliographystyle{IEEEtran}
\bibliography{IEEEabrv,IEEEexample}

\begin{IEEEbiography}[{\includegraphics[clip,width=1in,height=1.25in,keepaspectratio,bb = 0 0 200 100, draft, type=eps]{../examples/CV-image.png}}]
{Your Name} All about you and the what your interests are.
\end{IEEEbiography}

\begin{IEEEbiographynophoto}
{Coauthor}Same again for the co-author, but without photo\end{IEEEbiographynophoto}

\end{document}
